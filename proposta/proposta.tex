% ---------------------------------------------------------------------------- %

\documentclass[a4paper]{article}

\usepackage[utf8]{inputenc}
\usepackage[portuguese]{babel}
\usepackage{a4wide}
\usepackage{fancyvrb}
\usepackage{framed}
\usepackage{graphicx}
\usepackage{mathtools}
\usepackage{minted}
\usepackage{mdframed}
\usepackage{color}
\usepackage{amssymb}
\usepackage[titletoc,title]{appendix}
\usepackage[hidelinks]{hyperref}
\usepackage[notindex,nottoc,notlot,notlof]{tocbibind}
\usepackage{sectsty}

% ---------------------------------------------------------------------------- %

\addto\captionsportuguese{%
    \renewcommand\listoflistingscaption{Lista de Listagens}%
    \renewcommand\listingscaption{Listagem}%
}

% Utility macros:

\newcommand{\secref}[1]{Secção~\ref{#1}}
\newcommand{\anxref}[1]{Anexo~\ref{#1}}
\newcommand{\figref}[1]{Figura~\ref{#1}}
\newcommand{\tblref}[1]{Tabela~\ref{#1}}
\newcommand{\itemizedpar}[1]{\paragraph{\textbf{#1}}}

\newminted{gawk}{frame=leftline,xleftmargin=19pt,framesep=10pt}
\newminted{text}{frame=leftline,xleftmargin=19pt,framesep=10pt}

% Author notes:

\newcommand{\mynote}[3]{\fbox{\bfseries\sffamily\scriptsize#1}{\small$\blacktriangleright$\textsf{\emph{\color{#3}{#2}}}$\blacktriangleleft$}}
\newcommand{\luis}[1]{\mynote{luis}{#1}{red}}
\newcommand{\fabio}[1]{\mynote{fabio}{#1}{blue}}

% Uncomment the following lines to hide author notes:
% \renewcommand{\luis}[1]{\ignorespaces}
% \renewcommand{\fabio}[1]{\ignorespaces}

% ---------------------------------------------------------------------------- %


\title{
    Proposta para o Projecto \\
    Aprendizagem Automática I
    }

\author{
    % \ \ \ \ \ \ \ \ \ \ \ \ \ \ \ \ \ \ \ \ \ \ \ \ Grupo ??\ \ \ \ \ \ \ \ \ \ \ \ \ \ \ \ \ \ \ \ \ \ \ \  \and
    Luís Meruje (A) \and
    Fábio Fontes (A78650)
}

\date{Novembro 2018}

% ---------------------------------------------------------------------------- %

\begin{document}

% ---------------------------------------------------------------------------- %
% Colocar a imagem da universidade no topo
% Organizar a página.

\maketitle

% ---------------------------------------------------------------------------- %

\clearpage
\section*{Membros}
\label{sec:membros}

% ---------------------------------------------------------------------------- %

\clearpage
\section*{Proposta para o projecto}
\label{sec:proposta}


% ---------------------------------------------------------------------------- %

\subsection*{Descrição do Problema}
\label{subsec:descricao_problema}


% ---------------------------------------------------------------------------- %

\subsection*{Descrição do Conjunto de Dados}
\label{subsec:descricao_conjunto}

% ---------------------------------------------------------------------------- %

\subsection*{Supervisionado ou Não Supervisionado}
\label{subsec:sup_nsup}

O problema em mãos trata-se de um problema supervisionada, 

% ---------------------------------------------------------------------------- %

\subsection*{Regressão ou Classificação}
\label{subsec:regre_class}

Regressão

% ---------------------------------------------------------------------------- %

\subsection*{Comentários e/ou Preocupações}
\label{subsec:comentarios}



% ---------------------------------------------------------------------------- %

\end{document}

% ---------------------------------------------------------------------------- %
